
Managing computer infrastructures in a efficient way has been, since ever,
one of the major problems in the world of Information Technology. When the
scale of the deployment grows it starts to be difficult to organize and
manage efficiently all the resources. Moreover, fixing or preventing
issues is a difficult task to accomplish since it is usually not easy to
detect and understand where the problem is.

This thesis outlines the results of one year of work at the IT Department
of CERN, the European Organization for Nuclear Research. The aim of this
work is to analyse and study how to detect and prevent the most common
issues in large computing infrastructures: configuration and package
drifts. After a deep study of the infrastructure and the issues affecting
it, we proposed two software solutions in order to limit these problems.
This document describes in detail the results of these studies and how the
software developed affected the CERN's infrastructure.

