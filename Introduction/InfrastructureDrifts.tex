\section{Infrastructure Drifts}

We call \textit{Infrastructure Drift} the tendency of one or more machines
to have a different configuration compared with the majority of the
cluster. This behaviour, if not detected, might lead to dangerous and
unpredictable situations.

A drift can have different causes and it is not always easy to detect and
address, usually a manual intervention is required to solve the problem.
It is possible to define two different categories of Infrastructure
Drifts: Package and Configuration Drifts.

\subsection{Package Drifts}

A package drift happens when a machine in a cluster has different packages
compared to another, or a different version of the installed packages.
When this happens the risk is to have a cluster serving requests with two
different versions of the same software.

Ideally all the machines in a cluster should have the same package
configuration and receive updates at the same time. The idea is to start
serving the requests with a new version of the software all at the same
time. 

\subsection{Configuration Drifts}

A Configuration Drift happens when the configuration of servers becomes
more and more different as times goes on. Machine updates, manual updates,
local patches can compromise the configuration of a running system and
make it differ from what it should be. Tools like Puppet make sure than
a machine is created with a given configuration but during its lifecycle,
the server it will always tend to drift from the original configuration.

Service managers usually try to prevent this behaviour using two different
actions: running a configuration management system and doing it frequently
to keep machines updated. Otherwise they rebuild machines periodically, in
this way a machine is rebuilt before it can deviate from the mainline.
