
% Page head
%\rhead[\fancyplain{}{\bfseries
%INTRODUCTION}]{\fancyplain{}{\bfseries\thepage}}
%\lhead[\fancyplain{}{\bfseries\thepage}]{\fancyplain{}{\bfseries
%INTRODUCTION}}
% Add introduction to the index
%\addcontentsline{toc}{chapter}{Introduction}
%Questa \`e l'introduzione.
% Do not enumerate last page
%\clearpage{\pagestyle{empty}\cleardoublepage}

In the last years, the world of Information Technology has changed
dramatically mainly due to different approaches and methodologies on how
to organize work in a company. We have seen clearly that companies that do
not evolve their way of working suffer from the competition of others; if
a company does not innovate it has very high chances to fail its mission.

The situation nowadays is different: in the past companies were used to
follow a common philosophy, agreeing on similar technologies and
approaching problems in the same way. Right now we can easily see
competitors that use different methods to manage their development and
infrastructure. The way a company handles its infrastructure and manages
its organization can be the difference between a success or a complete
failure.

With the exponential growth of the number of users connected to the
internet we see day by day a constant interaction between the final user
with the product or a service that he is using. Our smart-phone, smart-tv
or laptop are constantly connected to the internet: we can play songs,
watch a movie or search for flights in a matter of seconds. In order to
have the satisfaction of the client, the service has to be available with
the highest uptime, providing a fast reply without letting the user wait
for the informations he needs. The technology surrounding the user has
grown dramatically in the last year and, since this trend is not going to
stop soon, the industry needs to evolve in order to serve the constant
growing number of requests. It is evident that the old approach to
computing does not scale very well: managing hundreds of servers manually
and, at the same time, evolve the service itself is not feasible in terms
of time and resources.

In order to solve this problem in the last years a growing number of
methodologies on how to manage software and infrastructure have been
studied. The aim is to automate as much as possible the server
infrastructure and the deployment of the code while guaranteeing an high
quality of the service provided.

The following chapter will describe how Agile Methodologies can improve
the work flow of a company with different techniques. Most of these have
been implemented successfully at CERN with an outstanding improvement in
terms of time to deployment and service stability.
