\section{Problems}

\subsection{Multiple service managers}

The current Puppet setup is not a traditional one since at CERN there are
several service managers maintaining different infrastructures but using
the same modules. Since most of the modules are shared across multiple
services we need to make sure that every change to those modules does not
affect others people using the same code.

Even if, as we said previously, the whole procedure is well tested and
usually allows different teams to detect problems before deployment, there
have been some issues in the past. In fact such a large number of git
repositories might become difficult to handle by hand: merging the wrong
branch, pushing to a different remote and other problems of this kind
happened in the past, mostly due to human error or inexperienced users.

\subsection{Deployment interval}

As stated in the previous sections, Puppet runs periodically in all the
servers with different interval depending on the result of the latest
report: if the latest report did not apply any change the interval is
increased and vice versa. This let most of the user test their code
triggering the Puppet agent from the cli, without waiting for the batch
run.

This means that when a change is merge to production it might need
sometime to be deployed in all the servers involved in the change.
Triggering the agent using the cli on multiple machine at the same time
might lead to an overload of the interactive cluster and it is definitely
not advised by the Configuration Team.

\subsection{Configuration and package drifts}

Dealing with such a large number of machines has his drawbacks in terms of
management. Even with the use of Puppet, MCollective and other tools to
manage machines in large scale environments it is not easy to keep track
of the status of all of them and it happens often that failures go
undetected.

It happens in fact that some machines are found in a state they are not
supposed to be, usually because there are other failures that prevent
Puppet from doing his job. This can have different causes that, most of
the time, require manual intervention.

What is difficult to do is spot those machines among all the others due to
the large number.
