\section{Modules Hostgroups and Environments}

The Puppet ecosystem at CERN is based on three different components that
we will describe in this section: Module, Hostgroups and Environments.

It's important to keep in mind that all those components are stored in
separate Git reposories in order to keep track of code changes and make
the configuration change process easy and trackable.

Every repository has at least two branches: master and qa. The code in
'master' will target all the machines in production, while the code in
'qa' will target all the QA machines, that are used to test the code
changes before merging it into production.

\subsection{Modules}

Puppet modules are used to configure a single service/software on a server
and they can be reused across different hostgroups. They provide several
options to customize the installation.

\subsection{Hostgroups}

At CERN we define Hostgroups as a group of servers that are part of the
same service. They contain the puppet code to configure and run a set of
machines, using modules to configure the software.

%EXPAND

\subsection{Environments}

Environments are collections of modules and hostgroups at different
development levels, based on their git history. They are defined in YAML
files living in a Git repository, Jens uses those files to create the
custom environments that are served to the puppet masters.

There are two golden environments:

\begin{itemize}
    \item 'production' - modules and hostgroup are cloned from the master branch of
their git repositories
    \item 'qa' - modules and hostgroups are cloned from the qa branch of their git
repositories
\end{itemize}

Moreover there are a set of custom defined environments mainly used for
testing purpose: they allow service managers to override some modules and
specify a different branch from the default one.

Those environments can be of two different types:

\begin{itemize} 
    \item Dynamic environments - Service managers can specify a default branch for
hostgroup and modules (usually master or qa) and override others. For
example:

\begin{lstlisting}[language=bash, frame=single]
default: qa
notifications: bob@cern.ch
overrides:
    modules:
        sssd: ai456
\end{lstlisting}

    \item Static environments (Snapshots) - These environments are completely static,
every module and hostgroup points to a specific git commit: the
configuration won't be updated without a manual intervention on the
environment file. For example:

\begin{lstlisting}[language=bash, frame=single]
notifications: bob@example.org
overrides:
    common:
        hieradata: commit/fb96070c9c77cc442ac60ba273768f547d376c17
        site: commit/fb96070c9c77cc442ac60ba273768f547d376c17
    hostgroups:
        adcmon: commit/8bf3ca9fe39a6f354dfc70377205ed806d6ae540
    [...]
\end{lstlisting}
\end{itemize}


