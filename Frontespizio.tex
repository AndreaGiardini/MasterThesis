%Il seguente frontespizio � inteso da usare con le seguenti opzioni nel preambolo:
% \documentclass[titlepage,twoside]{scrreprt}
% \usepackage[...]{classicthesis}
%
%NOTA BENE: credo che l'opzione "twoside" di scrreprt non sia necessaria, perch� se non messa dovrebbe
%			impaginare tutto con rilegatura (e margine maggiore) a sinistra. Nel dubbio, controllate.
%
%Per utilizzare il frontespizio, copiare questo file nella stessa directory del
%file principale da cui compilate la tesi e in quest'ultimo copiare dopo \begin{document}:
% %Il seguente frontespizio � inteso da usare con le seguenti opzioni nel preambolo:
% \documentclass[titlepage,twoside]{scrreprt}
% \usepackage[...]{classicthesis}
%
%NOTA BENE: credo che l'opzione "twoside" di scrreprt non sia necessaria, perch� se non messa dovrebbe
%			impaginare tutto con rilegatura (e margine maggiore) a sinistra. Nel dubbio, controllate.
%
%Per utilizzare il frontespizio, copiare questo file nella stessa directory del
%file principale da cui compilate la tesi e in quest'ultimo copiare dopo \begin{document}:
% %Il seguente frontespizio � inteso da usare con le seguenti opzioni nel preambolo:
% \documentclass[titlepage,twoside]{scrreprt}
% \usepackage[...]{classicthesis}
%
%NOTA BENE: credo che l'opzione "twoside" di scrreprt non sia necessaria, perch� se non messa dovrebbe
%			impaginare tutto con rilegatura (e margine maggiore) a sinistra. Nel dubbio, controllate.
%
%Per utilizzare il frontespizio, copiare questo file nella stessa directory del
%file principale da cui compilate la tesi e in quest'ultimo copiare dopo \begin{document}:
% %Il seguente frontespizio � inteso da usare con le seguenti opzioni nel preambolo:
% \documentclass[titlepage,twoside]{scrreprt}
% \usepackage[...]{classicthesis}
%
%NOTA BENE: credo che l'opzione "twoside" di scrreprt non sia necessaria, perch� se non messa dovrebbe
%			impaginare tutto con rilegatura (e margine maggiore) a sinistra. Nel dubbio, controllate.
%
%Per utilizzare il frontespizio, copiare questo file nella stessa directory del
%file principale da cui compilate la tesi e in quest'ultimo copiare dopo \begin{document}:
% \input{Frontespizio}
%
%Quindi, modificare i successivi parametri a proprio piacimento

%Il titolo della tesi
\newcommand{\myTitle}{Codifica di immagini RGB-D mediante caratteristiche locali e salienza}
%L'autore della tesi
\newcommand{\myAuthor}{Simone Laierno}
%La scuola di appartenenza (meglio se tutto in maiuscolo)
\newcommand{\mySchool}{SCUOLA DI INGEGNERIA E ARCHITETTURA}
%Il corso di laurea
\newcommand{\myCourse}{Ingegneria Informatica}
%La materia in cui si svolge la tesi
\newcommand{\mySubject}{Computer Vision and Image Processing M}
%Il relatore
\newcommand{\myCoordinator}{Prof. Luigi Di Stefano}
%Il correlatore, se presente. Se non � presente e si decommenta \advisorfalse, 
%si pu� commentare o cancellare la riga seguente
\newcommand{\myAdvisor}{Dr. Alioscia Petrelli}
%L'anno accademico
\newcommand{\myYear}{2014--2015}
%La sessione in cui ci si laureer�
\newcommand{\session}{II}

\newif\ifmaster
\mastertrue %commentare se � un corso di laurea triennale
%\masterfalse %commentare se � un corso di laurea magistrale

\newif\ifadvisor
%\advisortrue %commentare se non � presente un correlatore
\advisorfalse %commentare se � presente un correlatore

\ifmaster
	\newcommand{\degreeIn}{Corso di Laurea Magistrale in}
	\newcommand{\thesisIn}{Tesi di Laurea Magistrale in}
\else
	\newcommand{\degreeIn}{Corso di Laurea in}
	\newcommand{\thesisIn}{Tesi di Laurea in}
\fi

\begin{titlepage}

%\areaset[50pt]{390pt}{598pt}

\begin{center}
{{\Large{\textsc{Alma Mater Studiorum $\cdot$ Universit\`a di
Bologna}}}} \rule[0.1cm]{390pt}{0.1mm}
\rule[0.5cm]{390pt}{0.6mm}
{\small{\sc \mySchool \\
\degreeIn\ \myCourse }}
\end{center}
\vspace{15mm}
\begin{center}
\linespread{1.2}
\LARGE{\textsc{\myTitle}}
\\
\end{center}
\begin{center}
\vspace{19mm}
{\large{\sc \thesisIn \\ \mySubject}}
\end{center}
\vspace{40mm}
\par
\noindent
\begin{minipage}[t]{0.47\textwidth}
{\large{Relatore:\\
\sc \myCoordinator}}
\ifadvisor
	\vskip 8pt
	{\large{Correlatore:}\\
	{\sc \myAdvisor}}
\fi
\end{minipage}
\hfill
\begin{minipage}[t]{0.47\textwidth}\raggedleft
{\large{Presentata da:\\
\sc \myAuthor}}
\end{minipage}
\vspace{20mm}
\begin{center}
{\large{\sc \session\ sessione\\%inserire il numero della sessione in cui ci si laurea
Anno Accademico \myYear}}%inserire l'anno accademico a cui si � iscritti
\end{center}
\end{titlepage}

%
%Quindi, modificare i successivi parametri a proprio piacimento

%Il titolo della tesi
\newcommand{\myTitle}{Codifica di immagini RGB-D mediante caratteristiche locali e salienza}
%L'autore della tesi
\newcommand{\myAuthor}{Simone Laierno}
%La scuola di appartenenza (meglio se tutto in maiuscolo)
\newcommand{\mySchool}{SCUOLA DI INGEGNERIA E ARCHITETTURA}
%Il corso di laurea
\newcommand{\myCourse}{Ingegneria Informatica}
%La materia in cui si svolge la tesi
\newcommand{\mySubject}{Computer Vision and Image Processing M}
%Il relatore
\newcommand{\myCoordinator}{Prof. Luigi Di Stefano}
%Il correlatore, se presente. Se non � presente e si decommenta \advisorfalse, 
%si pu� commentare o cancellare la riga seguente
\newcommand{\myAdvisor}{Dr. Alioscia Petrelli}
%L'anno accademico
\newcommand{\myYear}{2014--2015}
%La sessione in cui ci si laureer�
\newcommand{\session}{II}

\newif\ifmaster
\mastertrue %commentare se � un corso di laurea triennale
%\masterfalse %commentare se � un corso di laurea magistrale

\newif\ifadvisor
%\advisortrue %commentare se non � presente un correlatore
\advisorfalse %commentare se � presente un correlatore

\ifmaster
	\newcommand{\degreeIn}{Corso di Laurea Magistrale in}
	\newcommand{\thesisIn}{Tesi di Laurea Magistrale in}
\else
	\newcommand{\degreeIn}{Corso di Laurea in}
	\newcommand{\thesisIn}{Tesi di Laurea in}
\fi

\begin{titlepage}

%\areaset[50pt]{390pt}{598pt}

\begin{center}
{{\Large{\textsc{Alma Mater Studiorum $\cdot$ Universit\`a di
Bologna}}}} \rule[0.1cm]{390pt}{0.1mm}
\rule[0.5cm]{390pt}{0.6mm}
{\small{\sc \mySchool \\
\degreeIn\ \myCourse }}
\end{center}
\vspace{15mm}
\begin{center}
\linespread{1.2}
\LARGE{\textsc{\myTitle}}
\\
\end{center}
\begin{center}
\vspace{19mm}
{\large{\sc \thesisIn \\ \mySubject}}
\end{center}
\vspace{40mm}
\par
\noindent
\begin{minipage}[t]{0.47\textwidth}
{\large{Relatore:\\
\sc \myCoordinator}}
\ifadvisor
	\vskip 8pt
	{\large{Correlatore:}\\
	{\sc \myAdvisor}}
\fi
\end{minipage}
\hfill
\begin{minipage}[t]{0.47\textwidth}\raggedleft
{\large{Presentata da:\\
\sc \myAuthor}}
\end{minipage}
\vspace{20mm}
\begin{center}
{\large{\sc \session\ sessione\\%inserire il numero della sessione in cui ci si laurea
Anno Accademico \myYear}}%inserire l'anno accademico a cui si � iscritti
\end{center}
\end{titlepage}

%
%Quindi, modificare i successivi parametri a proprio piacimento

%Il titolo della tesi
\newcommand{\myTitle}{Codifica di immagini RGB-D mediante caratteristiche locali e salienza}
%L'autore della tesi
\newcommand{\myAuthor}{Simone Laierno}
%La scuola di appartenenza (meglio se tutto in maiuscolo)
\newcommand{\mySchool}{SCUOLA DI INGEGNERIA E ARCHITETTURA}
%Il corso di laurea
\newcommand{\myCourse}{Ingegneria Informatica}
%La materia in cui si svolge la tesi
\newcommand{\mySubject}{Computer Vision and Image Processing M}
%Il relatore
\newcommand{\myCoordinator}{Prof. Luigi Di Stefano}
%Il correlatore, se presente. Se non � presente e si decommenta \advisorfalse, 
%si pu� commentare o cancellare la riga seguente
\newcommand{\myAdvisor}{Dr. Alioscia Petrelli}
%L'anno accademico
\newcommand{\myYear}{2014--2015}
%La sessione in cui ci si laureer�
\newcommand{\session}{II}

\newif\ifmaster
\mastertrue %commentare se � un corso di laurea triennale
%\masterfalse %commentare se � un corso di laurea magistrale

\newif\ifadvisor
%\advisortrue %commentare se non � presente un correlatore
\advisorfalse %commentare se � presente un correlatore

\ifmaster
	\newcommand{\degreeIn}{Corso di Laurea Magistrale in}
	\newcommand{\thesisIn}{Tesi di Laurea Magistrale in}
\else
	\newcommand{\degreeIn}{Corso di Laurea in}
	\newcommand{\thesisIn}{Tesi di Laurea in}
\fi

\begin{titlepage}

%\areaset[50pt]{390pt}{598pt}

\begin{center}
{{\Large{\textsc{Alma Mater Studiorum $\cdot$ Universit\`a di
Bologna}}}} \rule[0.1cm]{390pt}{0.1mm}
\rule[0.5cm]{390pt}{0.6mm}
{\small{\sc \mySchool \\
\degreeIn\ \myCourse }}
\end{center}
\vspace{15mm}
\begin{center}
\linespread{1.2}
\LARGE{\textsc{\myTitle}}
\\
\end{center}
\begin{center}
\vspace{19mm}
{\large{\sc \thesisIn \\ \mySubject}}
\end{center}
\vspace{40mm}
\par
\noindent
\begin{minipage}[t]{0.47\textwidth}
{\large{Relatore:\\
\sc \myCoordinator}}
\ifadvisor
	\vskip 8pt
	{\large{Correlatore:}\\
	{\sc \myAdvisor}}
\fi
\end{minipage}
\hfill
\begin{minipage}[t]{0.47\textwidth}\raggedleft
{\large{Presentata da:\\
\sc \myAuthor}}
\end{minipage}
\vspace{20mm}
\begin{center}
{\large{\sc \session\ sessione\\%inserire il numero della sessione in cui ci si laurea
Anno Accademico \myYear}}%inserire l'anno accademico a cui si � iscritti
\end{center}
\end{titlepage}

%
%Quindi, modificare i successivi parametri a proprio piacimento

%Il titolo della tesi
\newcommand{\myTitle}{Codifica di immagini RGB-D mediante caratteristiche locali e salienza}
%L'autore della tesi
\newcommand{\myAuthor}{Simone Laierno}
%La scuola di appartenenza (meglio se tutto in maiuscolo)
\newcommand{\mySchool}{SCUOLA DI INGEGNERIA E ARCHITETTURA}
%Il corso di laurea
\newcommand{\myCourse}{Ingegneria Informatica}
%La materia in cui si svolge la tesi
\newcommand{\mySubject}{Computer Vision and Image Processing M}
%Il relatore
\newcommand{\myCoordinator}{Prof. Luigi Di Stefano}
%Il correlatore, se presente. Se non � presente e si decommenta \advisorfalse, 
%si pu� commentare o cancellare la riga seguente
\newcommand{\myAdvisor}{Dr. Alioscia Petrelli}
%L'anno accademico
\newcommand{\myYear}{2014--2015}
%La sessione in cui ci si laureer�
\newcommand{\session}{II}

\newif\ifmaster
\mastertrue %commentare se � un corso di laurea triennale
%\masterfalse %commentare se � un corso di laurea magistrale

\newif\ifadvisor
%\advisortrue %commentare se non � presente un correlatore
\advisorfalse %commentare se � presente un correlatore

\ifmaster
	\newcommand{\degreeIn}{Corso di Laurea Magistrale in}
	\newcommand{\thesisIn}{Tesi di Laurea Magistrale in}
\else
	\newcommand{\degreeIn}{Corso di Laurea in}
	\newcommand{\thesisIn}{Tesi di Laurea in}
\fi

\begin{titlepage}

\areaset[50pt]{390pt}{598pt}

\begin{center}
{{\Large{\textsc{Alma Mater Studiorum $\cdot$ Universit\`a di
Bologna}}}} \rule[0.1cm]{390pt}{0.1mm}
\rule[0.5cm]{390pt}{0.6mm}
{\small{\sc \mySchool \\
\degreeIn\ \myCourse }}
\end{center}
\vspace{15mm}
\begin{center}
\linespread{1.2}
\LARGE{\textsc{\myTitle}}
\\
\end{center}
\begin{center}
\vspace{19mm}
{\large{\sc \thesisIn \\ \mySubject}}
\end{center}
\vspace{40mm}
\par
\noindent
\begin{minipage}[t]{0.47\textwidth}
{\large{Relatore:\\
\sc \myCoordinator}}
\ifadvisor
	\vskip 8pt
	{\large{Correlatore:}\\
	{\sc \myAdvisor}}
\fi
\end{minipage}
\hfill
\begin{minipage}[t]{0.47\textwidth}\raggedleft
{\large{Presentata da:\\
\sc \myAuthor}}
\end{minipage}
\vspace{20mm}
\begin{center}
{\large{\sc \session\ sessione\\%inserire il numero della sessione in cui ci si laurea
Anno Accademico \myYear}}%inserire l'anno accademico a cui si � iscritti
\end{center}
\end{titlepage}
