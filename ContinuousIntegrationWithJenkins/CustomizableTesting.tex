\section{Customizable Testing}

As we said in the previous sections, the idea is to have a set of tests
customizable for every Puppet module. In order to do this a new folder has
been introduced in \textit{code/manifests/tests}.

This folder contains all the tests and the configuration files that the
user can modify to customize his experience.

This folder, after his initialization, contains four files:

\begin{itemize}
  \item \textit{check.pp}

This file is a Puppet manifest. The module maintainer specifies here how
the machine to be tested needs to be. Since the aim is to test a single
module, this manifest should include the module itself with some basic
parameters.

Using this technique Jenkins can easily create machine customized on the
user needs.

  \item \textit{run\_tests.sh}

This bash script specifies which

  \item \textit{run\_tests\_external.sh} - A script to trigger tests
  outside the machine



  \item \textit{config.sh} - Contains some variables to customize the testing



\end{itemize}


