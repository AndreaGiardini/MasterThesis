\section{Infrastructure Overview}

Jenkins is one of the most popular Continuous Integration tools in the
open-source environment: it is used by companies to achieve different
goals, due to its highly-customizable nature. Moreover it is officially
supported by the configuration team so it looks like a good system to base
our CI on.

The idea is to keep Jenkins as light as possible: we will try to avoid to
create heavy jobs, that are difficult to debug in case of problems. Instead we
want to modularize our installation as much as possible creating small jobs,
linked with each other.

Moreover, we want to avoid to use the Jenkins interface, delegating most part
of the logic to scripts. Every Jenkins job needs to be small and the only
action that it has to do is trigger an external script, maintained in a Git
repository. At the beginning of each job, the Git repository
\textit{aici-servant-scripts} is cloned and the corresponding script is
triggered with the job's parameters.

Small jobs are then aggregated using the MultiJob plugin, which allows us to
chain more jobs together, passing parameters from one to the other.

The user is able to specify the parameters and the tests for his job inside the
same Git repository containing the puppet module, as it is described in the
section \textit{Customizable Testing}.

\subsection{Jenkins cluster}

The Jenkins cluster is made by a single master and four slave machines. All
those servers share an AFS folder between them to keep some common files like,
for example, the list of common settings and some temporary files for each job.

Using this configuration allows us to dynamically scale the cluster in case of
problem without having to restart Jenkins. If the cluster starts to be
overloaded we can build a new slave automatically, attach the network share and
let it join the cluster in a timely manner. In fact, since our work flow is
split in different sub-jobs it is highly possible that sub-jobs that are part
of the same macro work flow are executed on different slaves.

\subsection{Jobs description}

In this section we will describe all the jobs involved in the processing of
a CRM.

\subsubsection{Create-env - Create Environment}

This jobs accepts these parameters:

\begin{itemize}
  \item MODULE - Name of the module
  \item MERGETO - Branch of the module to merge to (destination\_branch)
  \item BRANCH - Branch of the module that contains the change (source\_branch)
  \item JIRA\_TICKET - The Jira ticket that keeps track of the CRM
\end{itemize}

When the job is triggered it creates a dynamic environment that will be
used to test the module. The \textit{MERGETO} variable represents the
default branch used in the environment file and the \textit{BRANCH}
variable is used as override of the \textit{MODULE}. The custom
environment is created using a tool called \textit{ai-create-environment}
and it is pushed to the \textit{it-puppet-environments} repository. After
pushing to the repository, the script queries Foreman periodically to make
sure that the environment has been created and that it is available before
proceeding. Once the environment is available the job finishes, returning
the name of the custom environment. Those environments have a name that
concatenates \textit{aici}, the name of the module and a random UUID (e.s.
aici-apache-u2r478c5).

This job is simple and aims to create a custom, private environment where
the CI can test the changes committed to \textit{BRANCH}. Creating
a custom environment is a mandatory step to be able to understand how the
change affects the final machine.

\subsubsection{Create-Verify - Builds a machine and tests it}

This job accepts these parameters:

\begin{itemize}
  \item ENVIRONMENT - Dynamic environment created by the \textit{create-env} job
  \item MODULE - Name of the module
  \item OS - Specifies on which operating system we want to run the tests
\end{itemize}

This is a MultiConfiguration job, it means that every time this job is
triggered it will run multiple times with different parameters. In our
case the \textit{ENVIRONMENT} and the \textit{MODULE} variables are passed
to the job, while \textit{OS} can assume three different values:
\textit{SLC5}, \textit{SLC6} and \textit{CC7}. The aim is to test our code
on the different architectures: the ones supported by the Computer
Center.

This job builds a virtual machine of the operating system specified in the
variable \textit{OS} and runs the tests. In order to speed-up the building
process, it has been agreed to keep three separate pools of machines ready
to be used, one for each operating system. Each pool contains a set of
machines ready to be used for the tests, they are built in advance, with
an empty manifest file, to make the testing process faster and avoid
errors during the spawning of the machines.

When the job is triggered:

\begin{itemize}
  \item A machine is picked from the appropriate pool
  \item The machine is moved to the appropriate hostgroup / environment
  \item The job waits for the machine to be built
  \item The job runs the tests and verifies status of the machine
\end{itemize}

This job can be customized by the module maintainer using the
configuration file described in the following section and it is successful
only if all the tests specified are successful. A sub-hostgroup of the
hostgroup \textit{it-puppet-hostgroup-testing} is created, including only
the module that we want to test, for example: \newline{}

\begin{lstlisting}[language=bash, frame=single]
  class hg_testing::apache { 
    include apache::tests::check
  }
\end{lstlisting}

Using this technique it is possible to configure a clean machine with
nothing more than the base modules (that are installed by default) and the
module that we want to test. Since we change both hostgroup and
environment, the machine that we are building will be configured with the
module we are testing and the change that we committed in the custom
branch.

Moreover, as we will see later, using the configuration files, it is
possible to specify on which Operating Systems the user wants to run its
tests on.

\subsubsection{Build-Spare - Fills the pools of spare machines}

This job accepts these parameters:

\begin{itemize}
  \item OS - Specifies which pool needs to be checked
\end{itemize}

This job is a MultiConfiguration job. Every time is triggered a script
checks the status of each pool and creates the corresponding machines if
it is not full. It is necessary to run this job to create new machines for
the tests, since after each testing the machine is destroyed.

The \textit{OS} variable, like the previous job, has three different
values representing the Operating Systems supported at CERN. Whenever this
job is triggered, three jobs are started, one for each possible value of
\textit{OS}. Of course it is possible to add more operating systems to the
support list simply adding a new value to the list.

While other jobs can have more than one instance at the same time (for
example, two destroy-vm) this job is meant to run sequentially since we
want to create one machine at the time and make sure that they are
correctly spawned and built. For this reason, even if this job is
triggered every time a machine is picked up for testing, there will always
be at maximum one build-spare job running.

This job is able to re-trigger itself automatically if the pools are not
full. If the maximum size of the pool is five, but the number of machines
currently present in the pool is three, the job will re-instantiate itself
two times in order to fill the pool.

\subsubsection{Destroy-Vm - Destroys the specified virtual machine}

This job accepts these parameters:

\begin{itemize}
  \item VM\_DESTROY\_NAME - Name of the machines to be deleted
\end{itemize}

Provided a list of machines to be deleted, this job removes all of them
from the system. This job is triggered at the end of the testing phase,
once all the results of the test are available and successful. If the
tests are not successful the server is not destroyed in order to have the
possibility of inspecting the faulty machine and see what happened.

\subsubsection{Delete-Environment - Removes the specified environment}

This job accepts these parameters:

\begin{itemize}
  \item ENVIRONMENT - Environment to be destroyed
\end{itemize}

Given the environment to be deleted, it deletes it from the system. In
fact, since the environments are basically mono-use, once the change has
been tested, there is no need to keep them.

\subsection{Integration with GitLab and Jira}

As described in the previous sections, it is important to keep the status
of the repository, ticket and configuration in sync. This means that the
CRM tickets that have been opened to start a new configuration change need
to be closed accordingly to the status of the CI.

Moreover, since CERN is migrating its Git infrastructure from gitolite to
GitLab its fundamental to find a way to integrate this component in the
work flow.

\subsection{AiDoctor}

Aidoctor is another component of the infrastructure that we need to take in
consideration when talking about the CRM process. It is nothing more than
a series of scripts that aim to automate some tasks that used to be done
manually in the past. In particular its duties are:

\begin{itemize}

  \item Send email for configuration changes

  Every time a new change is submitted to a module and it is merged in the QA
  branch all the service managers that use the specified module receive an
  email stating that a change is ongoing. In this way a service manager knows
  that it is necessary to monitor closely the QA servers for possible
  misbehaviours.

  \item Weekly CRM recap

  Since the Configuration Management team meets once a week with the others
  teams it is necessary to provide a list of the changes that are in QA
  and the changes that went in production during last week. Other service
  managers are then able to ask questions or clarifications directly to
  the team about configuration changes.

  \item Update ticket status and trigger Jenkins

  In some cases, when webhooks are not available, Aidoctor takes care of
  monitoring the status of tickets in Jira and updates them accordingly to
  their status. In particular this script is responsible for triggering
  a Jenkins build when the CRM ticket is opened and, if the configuration
  change passes the QA period, trigger a new build to move the change to
  production.

\end{itemize}

The authorization for Aidoctor works both with Kerberos tickets, that are
periodically renewed, or cookies.
