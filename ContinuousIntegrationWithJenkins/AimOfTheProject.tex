\section{Aim of the project}

The CRM process described in the previous sections works well for our
needs but it has one major flaw: it is not automated and it is easy to
make mistakes. In particular merging different branches and keeping the
Jira ticket in sync with the status of the deployment and updating the
ticket accordingly are usually source of problems. In the past we have
seen merges between wrong branches, or outdated tickets which are not
following the status of the repository. This leads obviously to confusion
in the organization of the deploy of new configurations since the process
is mostly manual.

Automating the CRM process and adding automated tests would be a great
improvement for the release of new features. The aim is to move all the
operations that were done manually before to an automated task, taking
care of all the operations.

Moreover, it would be nice to include automated tests customized by the
module maintainer on different operating systems. The tests needs to be
triggered from inside and outside the machine that needs to be tested. The
user should be able to specify how the machine to test needs to be built.

The system needs to be well integrated with all the different parts of the
system like Git, Jira and Gitlab using API.

The second part of this thesis will focus on all these aspects: we will
see how we managed to make all this possible using Jenkins, a Continuous
Integration tool.
