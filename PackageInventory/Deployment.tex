\section{Deployment}

Once we finished the first stage of development of PackageInventory we
decided to try it on some QA machines to spot some bugs before moving it
to the production environment. This operation was indeed very useful and
it gave us an opportunity to see how PackageInventory was able to scale
with an high number of machines.

We started in the beginning with a small number of machines, around twenty
servers part of the Lxplus QA cluster. During a couple of weeks we
continued fixing bugs and improving the Reporter and the Cli following the
guidelines of the final users. When we started to be confident about the
software that we built, we decided then to ship it in the production
Lxplus cluster with over two hundred machines.

This process has been repeated multiple times, every time we increased the
size of the deployment to see how PackageInventory was scaling and if the
ElasticSearch cluster was able to handle the load.

By the end of my contract at CERN, PackageInventory has been deployed in
several clusters of servers, including the Batch Cluster (the bigger
cluster at CERN) with more that two thousands machines. In total, the
number of hosts with PackageInventory installed are more than fifty five
hundreds of servers.

During this procedure of scaling and deployment we had of course to fine
tune the PackageInventory and the ElasticSearch cluster. We would like to
highlight in particular one incident than we had when we tried to deploy
PackageInventory on a large number of server at the same time (more than
one thousand machines installed and reported together). As we said
previously our software reports every time the difference in the packages
of an host compared to the previous run. During the first run however,
PackageInventory reports all the packages installed to the ElasticSearch
server, generating quite a large report. We can definitely say that, since
all the other reports are incremental, the first report of every host is
the heaviest one to process by the ElasticSearch cluster.

When we deployed PackageInventory to a large number of nodes we realized
that our ElasticSearch cluster was not handling the load properly, making
the cluster crash. On one hand we notices that it was not a problem of
resources, but more of a wrong configuration of ElasticSearch's
parameters. After some investigations we increased the number of open
files and sockets for the cluster and it started then to process all the
metrics correctly.

It is important to notice that, since we used Flume to guarantee
redundancy, no metrics have been lost. In fact Flume noticed immediately
that the cluster was not available and queued the metrics locally waiting
for the cluster to be online again.

% TODO talk about load split between ES and host
