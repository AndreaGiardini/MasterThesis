\section{Cli}

A Cli is provided to query the status of packages and hosts from the
user's machines. It uses PuppetDB and the Elasticsearch cluster to return
the results to the user. A Kibana dashboard is provided as well to
constantly show the incoming number of reports and the activity of Package
Inventory.

\subsection{Functions}

\begin{itemize}
  \item \textbf{Compare two or more hosts or machines}

  PackageInventory allows us to compare two or more machines. We are able to 
  query PuppetDB to get a list of hosts or we can specify the host names. For 
  example, if we want to compare all the hosts on a specific hostgroup:
  
  \begin{lstlisting}[frame=single]
  \$ pkginv -H \'bi/batch/gridworker/aishare/share\' -e production compare
  
  Processing hosts with:
  
       - Hostgroup: bi/batch/gridworker/aishare/share
       - Environment: production
  \end{lstlisting}

  \item \textbf{Package List}
  \item \textbf{Package History}
  \item \textbf{Package Status}
\end{itemize}

\subsection{Querying ElasticSearch}

The query language of ElasticSearch is substantially different from the
one we used to have with relational databases like MySQL\@. In fact
ElasticSearch does not have the concept of tables, keys or triggers,
instead it provides databases with different mappings. This means that
when a database is created is necessary to provide a description of the
metrics that we want to store in it.

Defining a correct mapping is extremely important when we are dealing with
ElasticSearch since the data will be stored and optimized for the type of
mapping that we defined. Changing the mapping of a database afterwards is
possible but, depending on the amount of data we already have, might be
extremely time-consuming since the database needs to re-index all the
metrics.



\subsection{Optimizations for large-scale queries}
