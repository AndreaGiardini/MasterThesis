\section{Project Description}

Following the growing number of machines installed in the data centres,
a solution for managing package drifts was needed. In the last years more
and more problems have been detected generated by configuration and
package drifting.

In particular outdated machines are always difficult to spot, and often it
happens that a set of machines are not aligned to the configuration due to
local problems. This may obviously lead to different and unexpected
behaviours. Moreover this situation may introduce security issues. Every
time a new vulnerability is released we want to make sure that all the
systems applied the patches correctly.

For this reason, in collaboration with the PES (Platform and Engineering
Services) team, we developed PackageInventory: a set of tools to query the
status of packages across multiple hosts. It has three main components:

\begin{itemize}
  \item Reporter
  \item Cli
  \item ElasticSearch Cluster
\end{itemize}

The first two elements have been developed from scratch while the
ElasticSearch server is used by the configuration team also to run other
services.

The aim of PackageInventory is to provide an easy way to spot differences
in servers configuration. Whenever a server behaves differently from the
others we can compare his configuration with the healthy machines of the
cluster and see if there is any difference in the set of packages. This is
particularly useful when users report an error in the cluster: tracking
the machine that served that particular user and comparing it with the
rest of the cluster can make us spot differences we were not aware of.
