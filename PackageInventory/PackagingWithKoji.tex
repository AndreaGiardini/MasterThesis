\section{Packaging with Koji}

At CERN we Koji to build and ship packages to repositories and,
ultimately, to all the severs. Koji is a tool used by the official Fedora
Project \cite{FedoraProject} for packaging software and organize the
packages in different repositories.

It is based on the concept of chroot environments: every package is build
in a new, clean environment in order to avoid possible contaminations or
ship unnecessary files.

Koji represents a main component of our infrastructure since it is
responsible for building the software before shipping it to all the
machines. A user is able to define a spec file which is a configuration
file that Koji uses to build the package.

In the spec file we define all the instructions to build our package
correctly: the package name, version, release, architecture, installation
instructions, etc.

Our spec file is able to generate two different packages from the same
repository, this allowed us to build at the same time the reporter and the
cli for PackageInventory and deploy them together. Inside the spec file we
specify all the files that are necessary to execute PackageInventory
correctly.

Specifying dependencies between packages (in our case, python libraries)
or configuration files is extremely easy. Moreover is fundamental to
specify configuration files inside the spec file, due to the fact that
they don't need to be replaced in case of a package upgrade.

Developers can build their packages using Koji and then assign them to the
appropriate repositories. Koji takes care of building the package
correctly, following the specifications of the spec file, for different
architectures.

The developer is allowed to build the package using Koji as many time as
he wants before actually publishing it to the final repository. This is
very useful when we would like to test a build: we can trigger a scratch
build on Koji, download the resulting package and test it in our test
environment. In this way we can be sure that our software has been
packaged correctly before deploying it in the production servers.

Once the package is ready to be deployed it has to follow the usual
Configuration Change Procedure (CRM). The package is deployed for the
first week in a QA repository; all the QA machines point to this
repository and they will receive the update at the next yum update. After
one week the package can be moved to the stable repository and then be
deployed to all the machines.

%TODO - Add code examples
