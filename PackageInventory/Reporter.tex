\section{Reporter}

PackageInventory Reporter is the first element that we are going to
analyze. It reports the status of the packages of a server to a central
ElasticSearch instance, creating a browsable metrics store.

The reporter is installed on every machine that we want to monitor and,
every time it runs, it checks if the list of packages has been updated
since the last run. If a new package has been installed or removed since
the last run, it sends a lists of updates to ElasticSearch using Flume. It
includes a Yum plugin to make the updates available as soon as possible:
whenever Yum is executed to modify the package configuration,
PackageInventory is executed consequentially.

It is possible to configure the reporter specifying some parameters using
the configuration file stored in \textit{/etc/packageinventory.conf}. This
file allows us to set the locations of all the log files, specify the
ElasticSearch hostname and the index. We decided to make the reporter
customizable in order to have the possibility in the future to organize
and store the metrics in different clusters.

Before starting to code we had to do some investigation in order to better
understand how to report the packages correctly and which tools to use. We
had to study how the package manager of CentOS works and learn how to
interact with it using our software. In particular we analysed two
components: 

\begin{itemize}
  \item \textbf{Yum}

  Yum \cite{YumWebsite} is the default package-management utility using
  the RPM package manager. It automatically handles updates and dependency
  management. Yum is based on software repositories which can be accessed
  remotely or locally.
  
  It is installed by default in all the Linux distributions used at CERN
  like RHEL, Scientific Linux and CentOs. In the latest distribution it
  has been replaced by a new package manager called DNF. It is
  consequentially important to understand how Yum works to accomplish our
  mission.

  \item \textbf{RpmDb}

  RpmDb \cite{RpmDbWebsite} is a database, used by Yum, containing all the
  information and metadata about the packages that are installed on the
  system. All the RPM-based distribution have a global RpmDb, listing the
  packages of the whole system.
  
  It includes several functions to access, read and write it in an atomic
  way: since it is a database we should prevent simultaneous writes using
  a locking mechanism. It uses Berkeley DB.

\end{itemize}

\subsection{Functions}

When PackageInventory runs it stores the list of all the RPMs currently
installed in the system in a logfile at
\textit{/var/log/rpmlistinventory.log}. During every run it loads the list
contained in this file and it compares it with the list of packages
currently installed in the system. In this way PackageInventory is able to
easily detect which packages changed and how.

A simple configuration file:

% TODO move all the code frames to minted

%\begin{minted}
\begin{lstlisting}[frame=single]
[PackageInventory]
LogFile: /var/log/rpmlistinventory.log
FlumeFile: /var/log/flume-pkginv.log
Tag: [PkgInventory]
ElServer: myelcluster.cern.ch
ElIndex: myelindex
HostName: myhost
\end{lstlisting}
%\end{minted}

We will describe here the different paramenters of the configuration file:

\begin{itemize}
  \item \textbf{LogFile}
  
  Specifies a log file containing the list of packages that has been
  detected in the last run on PackageInventory. The list is comprehensive
  of package name, version, epoch, architecture and a timestamp reporting
  the date of installation.
  
  \item \textbf{FlumeFile}

  This options contains the name of the file that Flume has to process and
  to send to ElasticSearch. It contains the list of modifications that
  needs to be stored in the database; Flume monitors this file for
  changes.

  \item \textbf{Tag}

  This element has been added in the past for testing purpose and could be
  removed in a future version of PackageInventory. It was used to
  differentiate the metrics submitted by our software from the others
  stored in the same cluster.

  \item \textbf{ElServer}

  Defines the hostname of the ElasticSearch server we want to send our
  metrics to since, as we said previously, we might want to send the data
  to different clusters.

  \item \textbf{ElIndex}

  Specifies the name of the ElasticSearch index.

  \item \textbf{HostName}

  Defines the hostname of the server to monitor.

\end{itemize}

Every time PackageInventory runs on a machine it checks that all these
parameters are specified correctly and then it reports the packages
installed using the following logic:

\begin{itemize}
    \item Loads the package list of the previous run (\textit{LogFile}).
    \item Queries RpmDb to get the list of packages that are currently
            installed.
    \item Detects if any package has been removed, installed or upgraded.
    \item The list of modified packages is logged to a second log file
            (\textit{FlumeFile}), which is then used by Flume to communicate
            to ElasticSearch.
    \item The list of current packages is written in \textit{LogFile}
\end{itemize}

This whole process is triggered, using a plugin that we wrote called
\textit{pkgreporter}, every time Yum runs. In this way the metric store
receives live updates from all the hosts in real time.

\subsubsection{Report changes with Flume}

The log file called \textit{FlumeFile} is then used by Flume to
communicate the changes to the ElasticSearch cluster using a \textit{Flume
TailSource}. With an interceptor Flume parses the file line by line
generating the corresponding metrics.

A caching mechanism is included in Flume: if the communication with the
cluster is not working no data is lost in case of downtime. Flume in fact
will store the metrics locally until the ElasticSearch cluster becomes
available again.

\subsection{Installation and configuration with Puppet}

To automate the installation of PackageInventory in multiple machines and
to integrate it with Puppet, we coded a Puppet module and we made it
available in the software repository of CERN.

In this way the installation and configuration of PackageInventory is
fully automated using Puppet. The final user has simply to add the
following code in the Puppet manifest of his host to let it start to
report packages:

\begin{lstlisting}[frame=single]
class {'pkginventory':
    el_host     => 'pkginv.cern.ch',
    el_index    => 'logs',
    el_cluster  => 'Package Inventory',
}
\end{lstlisting}

When the manifest will be applied it will install the correct packages for
the execution of PackageInventory and its dependencies, set the
configuration file accordingly, install the Yum plugin and create a new
Flume TailSource.
