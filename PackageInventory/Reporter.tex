\section{Reporter}

Packageinventory Reporter is the first element that we are going to
analyze. It reports the package status to a central ElasticSearch
instance, creating a browsable metrics store.

Every time it runs, it checks if the list of packages has been updated
since the last run and, if the answer is positive, it sends the lists of
updates to ElasticSearch using Flume. It includes a Yum plugin to make the
updates available as soon as possible: whenever yum is executed,
packageinventory is executed consequentially.

It is possible to configure the reporter specifying some paramenter using
a configuration file stored in \textit{/etc/packageinventory.conf}. This
file allows us to set the locations of all the logfiles, specify the
ElasticSearch hostname and the index.

\subsection{Yum}

Yum is the default package-management utility using the RPM package
manager. It automatically handles updates and dependency management. Yum
is based on software repositories which can be accessed remotely or
locally.

It is installed by default in all the linux distributions used at CERN
like RHEL, Scientific linux and CentOs. It is consequentially important to
understand how Yum works to accomplish our mission.

\subsection{RpmDb}

RpmDb is a database, used by Yum, containing all the information and
metadata about the packages that are installed on the system. All the
RPM-based distribution have a global RpmDb, listing the packages of the
whole system.

It includes several functions to access, read and write it in an atomic
way: since it is a database we should prevent simultaneous writes using
a locking mechanism. It uses Berkeley DB (BDB) as a database.

\subsection{Functions}

When PackageInventory runs it stores the list of all the RPMs currently
installed in the system in a logfile at
\textit{/var/log/rpmlistinventory.conf}. During every run it loads the
list contained in this file and it compares it with the list of packages
currently installed in the system. In this way PackageInventory is able to
easily detect which packages changed and how.

Every time it runs:

\begin{itemize}
    \item Checks the configuration file, making sure that none of the options
            is missing.
    \item Loads the package list of the previous run (\textit{LogFile}
            if present).
    \item Queries RpmDb to get the list of packages that are currently
            installed.
    \item Detects if any package has been removed, installed or upgraded.
    \item The list of modified packages is logged to a second logfile
            (\textit{FlumeFile}), which is then used by Flume to communicate
            to ElasticSearch.
    \item The list of current packages is written in \textit{Logfile}
\end{itemize}

This whole process is triggered, using a plugin called pkgreporter, every
time Yum runs. In this way the metric store receives live updates from all
the hosts in realtime.

\subsection{Report changes with Flume}

The logfile called \textit{FlumeFile} is then used by Flume to communicate
the changes to the ElasticSearch cluster using a \textit{Flume TailSource}
on the file generated by PackageInventory. With an interceptor Flume
parses the file line by line generating the corresponding metrics.

A caching mechanism is included in Flume: if the communication with the
cluster is not working no data is lost in case of downtime.
