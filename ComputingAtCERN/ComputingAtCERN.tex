This chapter will provide an overview of the hardware, software and tools
that are currently used to manage both CERN's datacentres.

The history of computing at CERN goes back to the early days of CERN
itself. Since the beginning CERN has been a centre for research and
testing on early models of computers. The collaboration with software and
technology companies like IBM \cite{IBMWebsite} and Red Hat
\cite{RedHatWebsite} has provided materials for the improvement of
computer architectures in the early stages of computing.

Looking at old documents of the CERN's datacentre it is easy to recognize
how the equipment used has followed progressively the evolution of
computing itself. Especially in the early days of computing we have seen
CERN at the lead of development in the world of Information Technology.
Today CERN is definitely the leader for what it concerns processing of
scientific data but bigger computing companies are now the lead of
research and development.

It is important to remark how the research projects started at CERN years
ago influenced our life: the most important invention has been the World
Wide Web by Tim Berners-Lee. In 1990 the first prototypes of an HTTP
server and client have been coded in the offices of CERN. The initial idea
was quite different from the web as we know it: the project was intended
to provide an easy way to exchange documents inside the organization,
nobody expected such a success. When Berners-Lee realized the importance
of the idea that he had (due to the growing interest by the surrounding
community) he decided to release the code and the specification of HTTP to
the public for free, determining the extreme success of his idea. In
addition to this, CERN has built the first prototype of capacitive
touch-screen.

In this chapter we will describe how the infrastructure of CERN evolved
during the years, how the Datacentres works, which links and machines
are deployed and the main tools used to organize the computing resources.
