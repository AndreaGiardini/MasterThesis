\section{Cloud Computing}

In the last years the number of resources needed by CERN grew
dramatically: more and more computing power was needed and the traditional
approach used to manage physical machines was not scalable anymore.
Deploying a machine use to take days due to the physical installation, and
the retirement of old machines was as well a complex process, usually
involving downtimes and problems for final users.

In 2012 CERN started to migrate its physical infrastructure to
a virtualization-based one: most of the machines are now virtualized using
OpenStack and KVM. During the last years CERN became a big contributor in
the open-source environment due to its active collaboration with the
development team of those two projects.

\subsection{OpenStack}

OpenStack \cite{OpenStackWebsite} is a free and open-source
cloud-computing software platform. It is composed by a set of different
projects working together using RESTFul APIs: each project has a single
function ( e.s. storage, computing, networking, etc. ) and, through
a single endpoint, the user is able to instantiate virtual machines
coordinating all of them.

Users are able to instantiate, delete, resize and edit virtual machines in
real-time: this makes the development cycle faster and shortens the
testing time.

OpenStack splits the physical servers in different \textit{cells} that are
used to identify machine with different characteristics, for example their
location (Meyrin or Wigner), their age of the equipment or the type of
power line they are connected to (critical or standard). Each team at CERN
has a defined quota, which is used in order to prevent misuse of the
resources: users can instantiate only a defined number of virtual machines
for each cell. Using this mechanism the cloud team is able to keep control
of the resources and quickly identify problems in the allocation of
resources. The use of cells has been proved to be very interesting and
efficient, especially to guarantee service redundancy. The most popular
use case consists in allocating resources in different locations to
guarantee redundancy in case of network or power failures.

\subsection{KVM}

KVM (Kernel-based Virtual Machine) \cite{KVMWebsite} is a virtualization
infrastructure for the Linux kernel that turns it into a hypervisor. It
requires a processor with hardware virtualization extension and it
supports a wide range of operating systems.

Using KVM it is possible to run multiple virtual machines on the same host
running an unmodified version of Linux or Windows. The kernel exposes to
each virtual machines a complete set of virtualized hardware.

% TODO add some more details
