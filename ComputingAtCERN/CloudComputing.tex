\section{Cloud Computing}

In the last years the number of resources needed by CERN grew
dramatically: more and more computing power was needed and the traditional
approach used to manage physical machines was not scalable anymore.
Deploying a machine use to take days due to the physical installation, and
the retirement of old machines was as well a complex process, usually
involving downtimes and problems for final users.

In 2012 CERN started to migrate its physical infrastructure to
a virtualization-based one: most of the machines are now virtualized using
OpenStack and KVM. During the last years CERN became a big contributor in
the open-source environment due to his active collaboration with the
development team of those two projects.

\subsection{OpenStack}

OpenStack is a free and open-source cloud-computing software platform.
It's composed by a set of different projects working together using
RESTFul APIs: each project has a single function ( e.s. storage,
computing, networking, etc. ) and, through a single endpoint, the user is
able to instantiate virtual machines coordinating all of them.

Users are able to instantiate, delete, resize and edit virtual machines in
real-time: this makes the development cycle faster and shortens the
testing time.

% TODO talk about tenants and cells, deployment in different datacenters
% based on cells

\subsection{KVM}

KVM (Kernel-based Virtual Machine) is a virtualization infrastructure for
the Linux kernel that turns it into a hypervisor. It requires a processor
with hardware virtualization extension and it supports a wide range of
operating systems.

Using KVM it is possible to run multiple virtual machines on the same host
running an unmodified version of Linux or Windows. The kernel exposes to
each virtual machines a complete set of virtualized hardware.
