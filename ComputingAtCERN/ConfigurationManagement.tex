\section{Configuration Management}

With the adoption of OpenStack, other technologies in CERN were 
re-evaluated in order to determine if they represented best practice or
if other alternatives were available. With the adoption of a virtualized
infrastructure, it was going to be important to scale configuration
management to the expected number of machines. It also remains the case
that OpenStack is a complex set of projects which require care in their
configuration. CERN choose a solution that would allow them to take
advantage of configuration code developed outside the organisation to
configure large numbers of machines, as well as OpenStack components.

\subsection{Configuration}

\subsubsection{Puppet}

Puppet \cite{PuppetWebsite} is an open-source configuration management
utility that allows service managers to describe the configuration of
their machines using its own declarative language. It supports Unix-like
and Windows machines.

The user describes the status of the machine and his resources in a file
called \textit{Puppet manifest}, using Puppet's declarative language or
Ruby DSL. Before applying the catalog to the machine it must be compiled
using the system's information: Facter \cite{FacterWebsite} retrieves from
the server all the information needed by Puppet to compile the catalog.
Puppet compiles the catalog using the information provided by Facter to
create a system-specific manifest that will be applied against the target
system.

Puppet describes the status of the system using a custom declarative
language which can be applied directly on the system or distributed to the
clients using a client/server paradigm. Service managers can specify their
configuration without the need to write system-specific commands, since
the description they provide using the Puppet manifest is operating system
independent.

\subsubsection{Foreman}

Foreman \cite{ForemanWebsite} is a tool to manage the lifecycle and the
provisioning of machines. It provides an easy to use web interface to
browse Puppet reports and manage machine's settings.

From Foreman the user is able to create and manage new machines assigning
them to a specified hostgroup/environment. When Puppet runs on a hosts it
queries Foreman's ENC to ask information about the machine and configures
it accordingly, compiling the Puppet code stored in Git.

\subsection{Other tools}

\subsubsection{Git}

Git is an open-source revision control system with support for distributed
non-linear workflows. Every Git working directory is a complete repository
with full history, independent of the network access.

In the latest years his popularity grew dramatically, mostly due to his
extreme elasticity, compatibility with existing protocols and support for
non-linear development.

\subsubsection{Jenkins}

Jenkins is an open-source continuous integration tool written in Java,
it's a server-based system running in a servlet container. It's
integration with most of the revision control systems and the multitude of
available plugins made its popularity grow year by year.

It is currently used by several open-source project to run unit and
functional tests on code written by contributors.

\subsubsection{Koji}

Koji is an RPM-based build system; it provides a flexible, secure and
reproducible way to build software. In combination with Git it provides
a fast way to package and distribute software across the servers.

When a developer is ready to release a new version of his software he can
use Koji to build and tag it to the correct repository. In this way at the
next distro-sync all the machines will receive the updates.

\subsubsection{Jens}

Jens is a python-toolkit developed at CERN to generate puppet environments
dynamically, based on some input metadata. It is used to give service
managers the elasticity they need to configure their services while
dealing with shared modules.

% TODO add how it works and what it parses

\subsubsection{Rundeck}

Rundeck automates operations and repetitive actions: it allows users to
define a workflow and a series of actions to do. Those workflows can be
scheduled or triggered by actions. In particular Rundeck became more and
more popular in the DevOps ecosystem due to the possibility to easily
integrate him with other components and software, making easy for final
users to define autonomously new workflows and procedures.

Tracking failures and reacting to them is one of the best features of
Rundeck: it is possible to define behaviours in case of job failures and
instruct Rundeck to react to different kind of problems. We can specify
some actions to roll-back a change, send a notification, or simply retry
the job after an interval, depending on the use-case. 

\subsubsection{MCollective}

MCollective is a framework for server orchestration: it allows user to run
actions on a multitude of hosts. It is used extensively when it is
necessary to deal with several machines at the same time: trigger
a command on multiple servers, get the status of a package or a service
across a cluster, etc..

It is well integrated with Puppet and Foreman, allowing us to trigger
actions on specified hostgroups/environments directly from the command
line just specifying the parameters of the servers we want to query.

\subsubsection{Roger}

Roger manages two important pieces of machine state. Whether you want
alarms switched on (or, more accurately, whether you want alarms to be
masked), and what the current state of the machine is. Optionally you can
then use those state transitions to take actions, such as to remove
a machine from a load balancer.

\subsubsection{TBag}

Due to the fact that service managers share Puppet modules and code, we
had to find a solution to provide secrets to hosts (like password,
certificates and keytab files): since many people have access to the
repositories, it is not secure to store secrets in there.

To solve this problem we developed TBag (Trusted Bag) a service that
allows users to define secrets at host/hostgroup level. The secrets can be
retrieved only by the specified host using a custom class in the puppet
manifest.

