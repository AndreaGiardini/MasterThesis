\section{Configuration Management}

Due to the adoption of OpenStack it was necessary to find a way to manage
and configure machines in an automated way: since it was not possible
anymore to configure all the machines manually CERN needed a solution to
set up new machines automatically. The Configuration Team provides a set
of tools and services that allow users to specify which tools and software
they need on their servers and get them configured in minutes.

\subsection{Puppet}

Puppet is an open-source configuration management utility that allows
service managers to describe the system configuration of their machines
using his own declarative language. It works with Unix-like and Windows
machines.

The user describes the status of the machine and his resources in a file
called "Puppet manifest", using Puppet's declarative language or Ruby DSL.
Before applying the catalog to the machine the catalog must be compiled
using the system's informations: Facter retrieves from the server all the
informations needed by Puppet to compile the catalog. Puppet compiles the
catalog using the informations provided by Facter to create
a system-specific manifest that will be applied against the target system.

Puppet describes the status of the system using a custom declarative
language which can be applied directly on the system or distributed to the
clients using a client/server paradigm. Service managers can specify their
configuration without the need to write system-specific commands, since
the description they provide using the Puppet manifest is operating system
independent.

\subsection{Foreman}

Foreman is a tool to manage the lifecycle and the provisioning of
machines. It provides an easy to use web interface to browse puppet
reports and manage machine's settings.

From Foreman the user is able to create and manage new machines assigning
them to a specified hostgroup/environment. When Puppet runs on a hosts it
queries Foreman's ENC to ask informations about the machine itself and
configures it accordingly.

\subsection{Git}

Git is an open-source revision control system with support for distributed
non-linear workflows. Every Git working directory is a complete full
repository with a complete history, independent of the network access.

In the latest years his popularity grew dramatically, mostly due to his
extreme elasticity, compatibility with existing protocols and support for
non-linear development.

\subsection{Jenkins}
\subsection{Koji}
\subsection{Jens}
\subsection{Rundeck}
\subsection{Mcollective}
\subsection{Roger}
\subsection{TBag}

