\section{Performance Report}

Both projects have been accepted by the users with enthusiasm and, during
the whole period at CERN, we continued to provide support, fixing bugs and
introducing new features on user request. In particular Package Inventory
has shown his usefulness in multiple occasions.

Since the first days after its deployment we had the opportunity to see
Package Inventory in action trying to deal multiple critical
vulnerabilities have been released online and needed to be patched. In
this scenario the tool that was previously used to compare packages and
hosts was not able to determine correctly the status of the system.
Marionette Collective ( or MCollective ) has been used for several years
at CERN in order to automate cluster of server an execute the same
operation over sets of machines. Unfortunately it fails whenever it needs
to handle with high number of machines with high load ( which is the most
common user case at CERN ). In the past multiple fixes have been applied
to make MCollective work even in this scenario but, even if some changes
made it more stable, it continued to be an unreliable solution for big
clusters.

For this reason we started building Package Inventory: to have the
possibility to process offline the package reports instead of, like
MCollective, running big queries live. This approach has been proved to be
the winning one: collecting the data in a database and processing them
offline has been proved to be a more reliable and faster approach.

As we can imagine Package Inventory sends a big amount of data ( list of
all the packages ) only during the first execution. In the following
executions just the difference with the previous run is sent, without
overloading the server. Moreover since the updates happen mostly
overnight, there is no necessity of live querying the machines, when we
could have a indexed database to handle all the data.

%TODO ci is slow cause people are lazy
