\section{Performance Report}

Both projects have been accepted by the users with enthusiasm and, during
the whole period at CERN, we continued to provide support, fixing bugs and
introducing new features on user request. In particular Package Inventory
has shown his usefulness in multiple occasions.

Since the first days after its deployment we had the opportunity to see
Package Inventory in action trying to deal whith multiple critical
vulnerabilities that have been released online and needed to be patched.
In this scenario the tool that was previously used to compare packages and
hosts was not able to determine correctly the status of the system.
MCollective has been used for several years at CERN in order to automate
cluster of server and execute the same operation over sets of machines.
Unfortunately it fails whenever it needs to handle a large number of
machines with high load (which is the most common user case at CERN). In
the past multiple fixes have been applied to make MCollective work even in
this scenario but, even if some changes made it more stable, it continued
to be an unreliable solution for big clusters.

For this reason we started building Package Inventory: to have the
possibility to process offline the package reports instead of, like
MCollective, running big queries live. This approach has been proved to be
the winning one: collecting the data in a database and processing them
offline has been proved to be a more reliable and faster approach.

As we can imagine Package Inventory sends a big amount of data (list of
all the packages) only during the first execution. In the following
executions just the difference with the previous run is sent, without
overloading the server. Moreover since the updates happen mostly
overnight, there is no necessity of live querying the machines, when we
could have a indexed database to handle all the data.

On the other hand users started approaching to the CI with a much slower
rate, even thou the configuration process has been documented in detail.
The cause of this was probably the lack of time for writing tests and use
causes that were actually meaningful for the platform. Using the CI while
leaving it not configured does not make much sense, other than automating
the merge action. On the other hand, writing tests that are meaningful
might not be an easy task to accomplish. 
