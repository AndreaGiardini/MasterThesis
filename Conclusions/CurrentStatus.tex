\section{Current Status}

Before deploying Package Inventory and the continuous integration platform
to production we ran extensive tests in order to prevent every possible
error to come up once shipped. This process has been proved to be
efficient and gave us good results. In particular, for Package Inventory,
we started deploying the software in small sets of machines at the same
time and tested the results for weeks. It is obviously difficult to
predict all the possible problems, especially at this scale: in fact, most
of the issues were related to the scalability of the whole process,
details are explained in the following section.

The deployment of the continuous integration platform has been planned
together with the migration of the Git backend from gitolite to Gitlab and
required a strict collaboration with the team responsible of Git. Since
the gitolite service was basically end of life it has been decided not to
take it in account and to focus on the Gitlab platform. Some test
repositories and test tickets have been created in order not to interfere
with the production repositories and platforms. Once the platform was
considered stable we started giving access to the CI to a small set of
users, with detailed instructions on how to use it. We spent some weeks
letting the users test the platform and collecting their opinion regarding
the usability of it and suggestions to improve it. This testing procedure
allowed the team to understand and find issues that were not clear to the
developers, especially regarding the intuitiveness of the platform. We
repeated this experiment multiple times with different groups of users,
implementing every time the suggestions of the previous group until we
were satisfied with the results and the feedbacks.

As of today, both Package Inventory and the continuous integration
platform are available for service managers and users. Documentation and
guides have been written in order to invite users to test both services
and provide feedbacks and suggestions to the configuration team.
