\section{Future Plans}

Both platform have substantial margin for improvements. Based on our
experience the users feedback has been a key player during our development
and helped us to improve the product in ways that we were not expecting.

Regarding Package Inventory we can definitely say that the software
reached a certain level of maturity and stability and it is safe for
deployment in production. The following improvements have been planned but
not yet implemented:

\begin{itemize}

  \item Scheduled checks with reporting on large clusters

  Since the monitoring action of Package Inventory needs to be done
  periodically it would be a good to integrate it with the current
  alarming system to provide notifications in case of errors. Periodical
  scans might identify problems before the users.

  \item Extend the cli interface

  The current Cli provides only some basic functionalities and can be
  extended to deliver a better experience for the user. As said in the
  previous sections, to deliver a better experience we need to let the
  user experiment with the current one and find out which operations are
  more common or more difficult to accomplish.

  \item Write libraries for different programming languages

  Package Inventory has been optimized for being used with the Cli but
  that is not the only way we want to interact with it: in fact it might
  be useful to build libraries to integrate it with the most common
  languages.

  \item Replace Flume with a more lightweight solution

  The idea of replacing Flume with a lighter alternative has been
  evaluated several times since Flume is Java-based and tends to be quite
  heavy on the machine when the number of logs increases. The candidate
  for substitution is LogStash but no timeline has been defined yet.

\end{itemize}

For CI

\begin{itemize}

  \item 

\end{itemize}
